%% LyX 2.2.3 created this file.  For more info, see http://www.lyx.org/.
%% Do not edit unless you really know what you are doing.
\documentclass[english]{article}
\usepackage[T1]{fontenc}
\usepackage[latin9]{inputenc}
\usepackage{geometry}
\geometry{verbose,tmargin=3cm,bmargin=3cm,lmargin=3cm,rmargin=3cm}
\setlength{\parindent}{0bp}
\usepackage{fancybox}
\usepackage{calc}
\usepackage{amsmath}
\usepackage{amsthm}
\usepackage{amssymb}

\makeatletter
%%%%%%%%%%%%%%%%%%%%%%%%%%%%%% Textclass specific LaTeX commands.
  \theoremstyle{definition}
  \newtheorem{defn}{\protect\definitionname}[section]
 \theoremstyle{definition}
 \newtheorem*{defn*}{\protect\definitionname}

%%%%%%%%%%%%%%%%%%%%%%%%%%%%%% User specified LaTeX commands.
\renewcommand{\labelenumi}{(\arabic{enumi})}
\renewcommand{\labelenumii}{(\arabic{enumi}.\arabic{enumii})}
\renewcommand{\labelenumiii}{(\arabic{enumi}.\arabic{enumii}.\arabic{enumiii})}
\renewcommand{\labelenumiv}{(\arabic{enumi}.\arabic{enumii}.\arabic{enumiii}.\arabic{enumiv})}

\makeatother

\usepackage{babel}
  \providecommand{\definitionname}{Definition}

\begin{document}

\title{\noindent Effiziente Algorithmen 1 - Zusammenfassung}

\author{\noindent Patrick Dammann}

\date{\noindent 21.05.2017}
\maketitle

\section{Probleme und Algorithmen}
\begin{quotation}
\noindent %
\noindent\shadowbox{\begin{minipage}[t]{1\columnwidth - 2\fboxsep - 2\fboxrule - \shadowsize}%
\textbf{Lineares kombinatorisches Optimierungsproblem}

\smallskip{}

Gegeben sind eine endliche Menge $E$, ein System von Teilmengen $\mathcal{I}\subseteq2^{E}$
(zul�ssige L�sungen) und eine Funktion $c:E\rightarrow\mathbb{R}$.
Es ist eine Menge $I^{\ast}\in\mathcal{I}$ zu bestimmen, so dass
${\displaystyle c(I^{\ast})=\sum_{e\in I^{\ast}}c(e)}$ minimal bzw.
maximal ist.%
\end{minipage}}

\medskip{}

\noindent %
\noindent\shadowbox{\begin{minipage}[t]{1\columnwidth - 2\fboxsep - 2\fboxrule - \shadowsize}%
\textbf{Euklidisches Traveling-Salesman-Problem}

\smallskip{}

Gegeben sind $n$ Punkte in der Euklidischen Ebene. Zu bestimmen ist
eine geschlossene Tour, die jeden Punkt genau einmal besucht und m�glichst
kurz ist.

$E=$ Menge der Kanten

$\mathcal{I}=$ Alle Mengen von Kanten, die eine Tour bilden%
\end{minipage}}

\medskip{}

\noindent %
\noindent\shadowbox{\begin{minipage}[t]{1\columnwidth - 2\fboxsep - 2\fboxrule - \shadowsize}%
\textbf{Euklidisches Matching-Problem}

\smallskip{}

Gegeben sind $n$ Punkte in der Euklidischen Ebene ($n$ gerade).
Zu bestimmen sind $\frac{n}{2}$ Linien, so dass jeder Punkt Endpunkt
genau einer Linie ist und die Summe der Linienl�ngen so klein wie
m�glich ist.

$E=$ Menge der Kanten

$\mathcal{I}=$ Alle Mengen von Kanten mit der Eigenschaft, dass jeder
Knoten zu genau einer der Kanten geh�rt.%
\end{minipage}}
\end{quotation}
\begin{description}
\item [{Einheitskosten-Modell}] Es werden nur die Schritte des Algorithmus
gez�hlt, die Zahlengr��en bleiben unber�cksichtigt.
\item [{Bit-Modell}] Die Laufzeit f�r eine arithmetische Operation ist
$M$, wobei $M$ die gr��te Kodierungsl�nge einer an dieser Operation
beteiligten Zahl ist.
\end{description}
\begin{defn}
\noindent Die Laufzeitfunktion $f_{A}:\mathbb{N}\rightarrow\mathcal{\mathbb{N}}$
ist in $\mathcal{O}(g)$ f�r eine Funktion $g:\mathbb{N}\rightarrow\mathcal{\mathbb{N}}$
falls es eine Konstante $c>0$ und $n_{o}\in\mathbb{N}$ gibt, so
dass $f_{A}\leq c\cdot g(n)$ f�r alle $n\geq n_{o}$.
\end{defn}
%
\begin{defn}
Ein Algorithmus hei�t \textbf{effizient} bzw. \textbf{polynomialer
Algorithmus}, wenn seine Laufzeit in $\mathcal{O}\left(n^{k}\right)$
liegt.

Ein Problem, das mit einem polynomialen Algorithmus gel�st werden
kann, hei�t \textbf{polynomiales Problem}.
\end{defn}
%
\pagebreak{}
\begin{defn*}
Ein Graph $G$ ist ein Tupel $G=(V,E)$\footnote{In der Vorlesung werden prim�r endliche, einfache, schleifenfreie
Graphen behandelt, die der einfachheit halber eine Notation ohne Inzendenzfunktion
nutzen k�nnen.} bestehend aus einer nicht-leeren Knotenmenge $V$ und einer Kantenmenge
$E$.

Ein Graph hei�t endlich, wenn $V$ und $E$ eindlich sind.

Wenn $e=\left\{ u,v\right\} \in E$ und $u,v\in V$, dann sind $u$
und $v$ Nachbarn bzw. adjazent, sind Endknoten von $e$ und werden
von $e$ verbunden.

Eine Kante $E\ni e=\{u,u\}$ hei�t Schleife.

Kanten mit $E\ni e=\{u,v\}=f\in E$ hei�en parallel oder mehrfach.

Ein Graph ohne Mehrfachkanten hei�t einfach.
\end{defn*}

\end{document}
